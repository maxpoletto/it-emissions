\documentclass[11pt]{article}
\usepackage[margin=0.7in]{geometry}
\usepackage[T1]{fontenc}
\usepackage{libertine}
\renewcommand{\sfdefault}{fvs} % Use as sans-serif
\renewcommand{\ttdefault}{fvm} % Use as monospace
\renewcommand{\ttfamily}{\small\fontfamily{\ttdefault}\selectfont} % Slightly smaller monospace font
\newcommand{\assumption}[1]{{#1}}
%\usepackage{helvet} % Helvetica font for sans-serif
%\usepackage{courier} % Courier font for monospace
\usepackage{hyperref}
\usepackage[backend=biber,style=authoryear,autocite=inline]{biblatex}
\usepackage{titlesec}
\usepackage{enumitem}
\usepackage{titling}

\titleformat{\section}{\sffamily\Large}{\thesection}{1em}{}
\titleformat{\subsection}{\sffamily\large}{\thesubsection}{1em}{}
\titleformat{\subsubsection}{\sffamily\normalsize}{\thesubsubsection}{1em}{}
\setlength{\parindent}{0pt}
\setlength{\parskip}{0.5em}

\addbibresource{ref.bib}
\title{Modeling the impact of IT- and AI-related emissions in a university setting.}
\author{Massimiliano Poletto, Nils Handler}

\begin{document}

\maketitle

\section{Abstract}
Information technology (IT) plays major role in society and it has a non-negligible carbon footprint: for example, at the University of Zürich, it accounted for over $6\%$ of total emissions in 2024. At the same time, artificial intelligence (AI) applications, and especially large language models (LLMs), are being adopted at historically high rates. Their growth and relatively high energy requirements are expected to increase the share of global electricity consumption for data centers from $\approx 1\%$ in 2024 to $\approx 3\%$ by 2030. Therefore, an assessment of IT and AI emissions is a necessary component of any net-zero roadmap. As part of the University of Zürich's own net-zero efforts, we have attempted to quantify and break down the impact of IT and AI on the university's carbon footprint. In this paper we present a simple interactive emissions model that approximates operational and lifecycle emissions of personal devices, university-owned computing and network infrastructure, cloud computing, and LLMs. Our analysis shows that IT infrastructure's embodied emissions are a dominant component of overall IT emissions, that operational impacts are comparatively modest, and that LLM adoption, even under aggressive growth scenarios, is unlikely to comprise more than $3-5\%$ of overall emissions.

\section{Introduction}

approximately $6.3\%$ of total emissions in 2024\cite{uzh:sustainability:report},

artificial intelligence (AI) applications are being adopted at historically high rates exceeding those of the adoption of the personal computer and the internet\cite{bick:ai:adoption}.

\section{Modeling methodology}

\section{Results}

\section{Related work}

\section{Conclusion}

\section{References}

\printbibliography

\end{document}
