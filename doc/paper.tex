\documentclass[11pt]{article}
\usepackage[margin=0.7in]{geometry}
\usepackage[T1]{fontenc}
\usepackage{libertine}
\renewcommand{\sfdefault}{fvs} % Use as sans-serif
\renewcommand{\ttdefault}{fvm} % Use as monospace
\renewcommand{\ttfamily}{\small\fontfamily{\ttdefault}\selectfont} % Slightly smaller monospace font
\newcommand{\assumption}[1]{{#1}}
\newcommand{\eg}{{\em e.g.}}
%\usepackage{helvet} % Helvetica font for sans-serif
%\usepackage{courier} % Courier font for monospace
\usepackage{hyperref}
\hypersetup{
    hidelinks
}
\usepackage[style=numeric]{biblatex}
\usepackage{titlesec}
\usepackage{enumitem}
\usepackage{titling}

\titleformat{\section}{\sffamily\Large}{\thesection}{1em}{}
\titleformat{\subsection}{\sffamily\large}{\thesubsection}{1em}{}
\titleformat{\subsubsection}{\sffamily\normalsize}{\thesubsubsection}{1em}{}
\setlength{\parindent}{0pt}
\setlength{\parskip}{0.5em}

\addbibresource{ref.bib}
\title{Modeling the impact of IT- and AI-related emissions in a university setting}
\author{Massimiliano Poletto\thanks{Independent consultant, max.poletto@gmail.com}, Nils Handler\thanks{University of Zürich, nils.handler@uzh.ch. Contact author.}}

\begin{document}

\maketitle

\section*{Abstract}
Information technology (IT) plays a major role in society and has non-negligible environmental impact, accounting for over 5\% of global electricity consumption and about 2\% of global carbon emissions. Moreover, artificial intelligence (AI) applications, especially large language models (LLMs), are being adopted at historically high rates. Their growth is expected to increase the share of global electricity consumption for data centers from $\sim 1\%$ in 2024 to $\sim 3\%$ by 2030. Therefore, an assessment of IT and AI emissions is a necessary component of any net-zero roadmap. As part of the University of Zürich's own net-zero efforts, we have attempted to quantify and analyze the impact of IT and AI on the university's carbon footprint. We present a simple interactive model of IT emissions on a university campus. The model approximates operational and lifecycle emissions of personal devices, university-owned computing and network infrastructure, cloud computing, and LLMs. The model is intended to provide university decision makers with emissions estimates under different scenarios, with sufficient accuracy to support effective decision-making. Our analysis shows that the embodied emissions of IT infrastructure and personal devices are a major component of overall IT emissions, that operational impacts are comparatively modest, and that LLM adoption, even under aggressive growth scenarios, is unlikely to comprise more than $5\%$ of overall IT-related emissions.

\section{Introduction}

Information technology accounts for over 5\% of global electricity consumption and at least 1.7\% of global carbon emissions~\cite{wb:itu:ict}. At the University of Zürich (UZH), which with over 38,000 students and staff is the largest Swiss university, IT in 2024 accounted for about 1.5 Mt CO$_2$e (megatons of carbon dioxide equivalent) of greenhouse gas (GHG) emissions, approximately 6.3\% of total university emissions in 2024~\cite{uzh:sustainability:report}. The cited report takes into account university-owned workstations, servers, and audiovisual equipment, but does include other IT-related activities such as networking, cloud computing, or the growing use of artificial intelligence, and in particular the adoption of generative AI and large language models (LLMs).

AI applications are being adopted at historically high rates, more quickly than the personal computer or the internet were adopted in the past~\cite{bick:ai:adoption}. Enterprise AI adoption has accelerated, with generative AI tools seeing widespread deployment across industries within months of their release~\cite{mckinsey2024ai}. In July 2025, ChatGPT received 2.5 billion queries per day, up from 1 billion in December 2024, and within an order of magnitude of Google's approximately 15 billion queries per day~\cite{techcrunch:chatgpt}. AI growth is currently projected to increase the share of global electricity consumption for data centers from $\sim 1\%$ in 2024 to $\sim 3\%$ by 2030~\cite{iea:ai:energy}.

The University of Zürich has set itself the target of becoming climate-neutral by 2030 within the categories of energy, transportation, food, and waste, with additional goals in 2035 and beyond~\cite{uzh:sustainability:report}. Given the current sizable contribution of IT to the university's emissions, and the potential for growth through AI adoption in the coming years, we felt it was necessary to understand IT emissions and their growth trajectories in more detail.

This paper presents a simple interactive model of IT emissions on a university campus. We developed it in the specific context of the University of Zürich, but believe that it is parameterizable and general enough to apply to other similar university campuses or large corporate facilities.

We seek to answer three broad questions:
\begin{itemize}
  \item What are the main contributors to the IT emissions of a university?
  \item Could migrating some on-campus IT services to commercial cloud providers decrease the overall campus footprint?
  \item What is the likely impact of the surge in AI adoption? In particular, will the emissions impact of AI be so large as to make other sources of IT emissions negligible?
\end{itemize}

Our model sets relatively broad system boundaries. It takes into account university-owned equipment but also university services hosted by commercial cloud providers and university use of commercial generative AI products. We have also parameterized it to account for student devices not owned by the university, since they are key to student work and constitute the bulk of the university's end-user devices. We attempt to model devices' full lifecycle impact, including manufacture and disposal. 

Experts rightly point to the many potential benefits of AI in the areas of climate change mitigation and adapation~\cite{climate:change:ai}. On a university campus, these might include more efficient building heating and cooling, improved transportation logistics, etc. However, those benefits are still to come and difficult to predict. In this paper, we consider only the {\em direct} impacts of AI in terms of increased electricity consumption and the resulting emissions.

The rest of this paper is organized as follows. Section~\ref{sec:model:methodology} presents the general modeling framework, including system boundaries and assumptions. Section~\ref{sec:model:details} justifies some of the specific parameter values used in the model. Section~\ref{sec:results} describes results and explores different scenarios and assumptions. Section~\ref{sec:recommendations} offers recommendations to university IT administrators. Section~\ref{sec:related:work} describes related work in this area.

\section{Modeling methodology}
\label{sec:model:methodology}

Our aim is to model energy use from university IT and AI broadly, without restricting ourselves to university-owned devices, specific IT departments, etc. At the same time, we want to be able to distinguish the contributions of different factors, such as student-owned devices or LLMs, and to compare the model's output against existing university data points (energy bills, etc.).

\subsection{System boundaries}

We define the boundaries of our model as follows.

\subsubsection*{Population}

We include IT-related activity from all members of the UZH community: faculty, staff, and students.

\subsubsection*{Devices}

Student-owned devices, such as laptops and smartphones, are in scope, since they are the primary means by which students access university IT services and do their coursework. Given the duration of studies, we include personal devices' embodied emissions in the model.

University-owned devices are in scope. We have accurate numbers for staff workstations, classroom computers, network access points, and audiovisual devices managed by the central IT department. We can only estimate other devices, such as computers managed by individual research groups or networking equipment. As before, we include these devices' embodied emissions in the model.

\subsubsection*{Cloud and supercomputing facilities}

UZH at present does not make significant use of commercial cloud compute, but the model allows simulating the shift of some fraction of the server load to cloud VMs.

Due to a lack of data we do not attempt to model emissions from application-level cloud services such as Microsoft 365 or Google Workspace. This omission leads to undercounting total IT energy use but to overestimating the potential relative impact of AI.

UZH faculties do make use of shared off-campus supercomputing facilities, primarily the Swiss National Supercomputing Center, CSCS~\cite{cscs}. We estimate the impact of these workloads based on available gross peak load numbers. Due to lack of data, we do not include the embodied emissions of these facilities.

\subsubsection*{Large language models}

We estimate inference-time emissions of current large language models based on the literature. It is harder to come by good estimates of training-time emissions and embodied emissions of the hardware, but we attempt to do so nonetheless in order to not risk undercounting the impact of LLMs.

\subsubsection*{Summary}

This table summarizes what is in scope in the current model. The main gap is emissions of cloud services such as Microsoft 365 or Goole Workspace.

\begin{center}
  \begin{tabular}{|l|c|c|}
  \hline
  \textbf{Device or service} & \textbf{Operation} & \textbf{Lifecycle} \\ \hline
  Student laptops \& phones & yes & yes \\
  University workstations, access points, etc. & yes & yes \\
  Cloud servers & yes & yes \\
  Cloud services & no & no \\
  Supercomputing facilities & yes & no \\
  LLMs & yes & yes \\ \hline
  \end{tabular}
  \label{tab:scopes}
  \end{center}

\subsection{Calculation approach}

\subsubsection*{Devices}

We model campus device emissions bottom-up. We first identify categories of devices (phones, laptops, workstations, monitors, routers, etc.). For each device type, we research a range of common models, then use manufacturer data sheets or other available documentation to identify operational power draw (in watts) as well as embodied emissions related to manufacture and disposal (in kg CO$_2$e) per device. Based on this data, we model the power draw and embodied emissions of an ``exemplar'' device of each type.

For each device type we also estimate (1) a count (\eg, a laptop for every student and staff), (2) an estimated annual duty cycle (\eg, laptops used 8 hours per day, 200 days per year), and (3) a useful lifetime (in years).

Given all this, for each device type we can straightforwardly obtain an annual energy use (kWh) and annual embodied emissions (embodied emissions amortized over useful lifetime) (kg CO$_2$e).

We then estimate the local grid's emissions intensity (g CO$_2$e / kWh) to convert the energy use to emissions and obtain a combined total annual emissions number. We obtain emissions intensity data from Electricity Maps~\cite{electricitymaps}.

In addition to this simple linear mode, the model can also operate stochastically, instantiating devices from a distribution. However, we have no good data on the precise form of such a distribution, so we limit our analysis to the linear mode.

\subsubsection*{Cloud and supercomputing facilities}

We attempt to model a hypothetical shift of some workloads from servers in an on-premise data center to a commercial cloud. The parameters under consideration include:

\begin{itemize}
  \item On-premise power usage effectiveness (PUE), the ratio of total data center power (including cooling) to power used for computation.
  \item Cloud data center PUE and grid emissions intensity (since the cloud data center may be located in a different region than the on-premise data center).
  \item An average conversion ratio of on-premise servers to cloud servers.
\end{itemize}

We measure the impact of cloud servers in terms of annual operational emissions and amortized emissions due to manufacturing and disposal, in units of kg CO$_2$e / year.

We do not attempt to model application-level cloud services.

For supercomputing facilities (in particular, CSCS) we rely on annual statistics of total power consumption for peak compute load and make assumptions about duty cycle. We do not have reliable data about the embodied emissions of the facility, but suspect that the fraction of these emissions that is attributable to UZH use is negligible relative to all other UZH emissions.

\subsubsection*{Large language models}

For LLMs, we estimate operational energy use (Wh per query) using current estimates in the literature of state-of-the-art models~\cite{devries2023growing,epoch2025howmuchenergydoeschatgptuse}.

We consider the training overhead and the embodied emissions of the servers and GPUs that the model runs as a fixed cost that is amortized across all queries to a model. As a result, in practice, this overhead is constantly shrinking. We incorporate it into the model as a configurable percentage overhead of the operational energy use.

\subsection{Parameters and assumptions}

The model's outputs are affected by several factors, and where we were unable to find good data we made assumptions that we considered reasonable. All of these parameters and assumptions are parameterizable by users of the model. We list these sources of uncertainty below.

\begin{description}
  \item[Cloud data center PUE] As of 2025, industry average power usage effectiveness is 1.56, with Google averaging 1.09 across its fleet~\cite{google:datacenter:efficiency} and Microsoft ranging from 1.12 to 1.30~\cite{microsoft:datacenter:efficiency}. We choose a PUE of 1.2 as representative of a state-of-the-art hyperscaler data center.
  \item[On-premise data center PUE] Based on limited historical power data, we assume an on-premise PUE of 2.0, attributable to inefficient cooling.
  \item[Duty cycle] We make some unsubstantiated but plausible assumptions about the number of days that different devices are used (\eg, end-user devices are not used every day) and their duty cycle throughut the day.
  \item[Electricity emissions intensity for on-premise data centers] For Switzerland we compute average hourly emissions intensity over 2024 based on data from Electricity Maps~\cite{electricitymaps} ($\sim 50$ g / kWh on average).
  \item[Electricity emissions intensity for cloud data centers and LLMs] We use the European Union's average emissions intensity of electricity generation in 2023, 207 g CO$_2$e / kWh~\cite{eea:emissions:2025}.
  \item[] 
\end{description}



\section{Model details}
\label{sec:model:details}

\subsection{Campus emissions}

We instantiate as many devices as specified in the UZH spreadsheet, plus enough laptops and phones
to cover the rest of the university population.

For each device instance, we pick embodied emissions, power draw, and duty cycles either
deterministically or by sampling from the appropriate distribution.

Combining power draw and duty cycle estimates with ground-truth data about the Swiss electricity mix,
we estimate annual operational emissions and add them to the embodied emissions scaled by the device
type's expected lifetime.

For on-campus data centers, for which presumably there are electricity bills or other measures of
overall energy use, we would like to use that data to evaluate our model.

A possible future refinement will be to model the duty cycle as a function of time of day
and incorporate finer-grained electricity mix data.

\subsection{Cloud emissions}

We model cloud emissions based on the assumptions and data described in
Section~\ref{sec:cloud_data_centers}. We suggest working with UZH's cloud vendor to obtain
more accurate data.

\subsection{LLM emissions}

We model LLM emissions by sampling from the per-capita daily query distribution and computing
the carbon intensity of each query based on the average European electricity mix. While there
are many uncertainties about the level of LLM adoption, there appears to be reasonable consensus
in the literature about the energy consumption of each query.

\subsection{Embodied emissions}

For several device types, especially end-user devices such as phones and laptops, embodied emissions
comprise a majority of lifetime emissions. However, the exact amount of these emissions varies
widely between different sources. Our approach has been to find several studies and industry references,
and then estimate means and standard deviations from those that seemed most comprehensive and reliable
(for example, Dell's Product Carbon Footprint reports about its servers).

\subsubsection{Laptops}

\begin{center}
\begin{tabular}{|r|l|}
\hline
\textbf{Production \& Disposal} & \\
\textbf{(kg CO$_2$e)} & \textbf{Source} \\ \hline
121 & \textcite{ecoinvent}  \\ \hline
104 & \textcite{teehan2013} (low) \\ \hline
338 & \textcite{teehan2013} (high) \\ \hline
244 & \textcite{rarecoil} \\ \hline
202 & \textcite{unctadder2024} \\ \hline
\end{tabular}
\label{tab:embodied_emissions:laptops}
\end{center}

The Ecoinvent numbers appear to be outliers on the low end.
The most comprehensive data comes from \textcite{rarecoil}, which lists manufacturer
information for over 90 popular devices. Mean supply chain emissions are
244 kg CO$_2$e with a standard deviation of 128 kg.
\assumption{We model laptop supply chain emissions using a truncated normal distribution with those characteristics.}

\subsubsection{Desktops}

\begin{center}
\begin{tabular}{|r|l|}
\hline
\textbf{Production \& Disposal} & \\
\textbf{(kg CO$_2$e)} & \textbf{Source} \\ \hline
238 & \textcite{ecoinvent} \\ \hline
303 & \textcite{teehan2013} \\ \hline
403 & \textcite{unctadder2024} \\ \hline
289 & \textcite{dellpcf} \\ \hline
277 & \textcite{boavizta} \\ \hline
\end{tabular}
\label{tab:embodied_emissions:desktops}
\end{center}

Most estimates are in the range of 250 - 400 kg CO$_2$e. The most detailed data comes from
Dell's Product Carbon Footprints (\cite{dellpcf}), which they publish for each of their products.
A sample of 15 desktops has mean supply chain emissions of 289 kg CO$_2$e with $\sigma = 80$ kg.
\assumption{We model desktop supply chain emissions using these statistics.}

\subsubsection{Servers}

\begin{center}
\begin{tabular}{|r|l|}
\hline
\textbf{Production \& Disposal} & \\
\textbf{(kg CO$_2$e)} & \textbf{Source} \\ \hline
383 & \textcite{teehan2013} \\ \hline
1252 & \textcite{davy2021} (Dell) \\ \hline
1912 & \textcite{davy2021} (EC2) \\ \hline
899 & \textcite{boavizta} (medium server)\\ \hline
\end{tabular}
\label{tab:embodied_emissions:servers}
\end{center}

\textcite{davy2021} reports data for many Dell servers and estimates for different types of EC2 instances.
The former have mean supply chain emissions of 1252 kg CO$_2$e with $\sigma = 330$ kg, the latter
1912 kg with $\sigma = 885$ kg.
\assumption{We model server supply chain emissions using the statistics corresponding to Dell servers.}

\subsubsection{Smartphones}

\begin{center}
\begin{tabular}{|r|l|}
\hline
\textbf{Production \& Disposal} & \\
\textbf{(kg CO$_2$e)} & \textbf{Source} \\ \hline
68 & \textcite{googlepixel8} (Pixel 8)\\ \hline
54 & \textcite{appleiphone13} (iPhone 13)\\ \hline
50 & \textcite{unctadder2024} \\ \hline
50 & \textcite{lovehagen2023} \\ \hline
\end{tabular}
\label{tab:embodied_emissions:phones}
\end{center}

\assumption{We model phone supply chain emissions using mean 50 kg CO$_2$e and $\sigma = 10$ kg.}

\subsubsection{Other devices}

We found less data for other types of devices. For now, we model them with averages based
on a handful of sources.

\begin{description}
    \item[Computer monitors] 344 kg CO$_2$e (\textcite{teehan2013}, \textcite{dellpcf})
    \item[Conference room displays] 753 kg CO$_2$e (scaling monitor by $(40''/27'')^2$)
    \item[Printer/copier stations] 1167 kg CO$_2$e (\textcite{ecoinvent})
    \item[Network equipment] We did not find reliable lifecycle assessments for routers and switches,
      and have no information about UZH network architecture. The vast majority ($80-95\%$) of
      lifecycle emissions of network gear stem from usage (\textcite{cisco2024}, \textcite{jacob2023}).
      For now, we model network equipment as a 5\% overhead on servers (1 router or switch for 20 servers).
      (A typical data center rack contains 42 1U servers and one switch or router.)
\end{description}

\subsection{Operational emissions}

For on-premise equipment, we would like to model the Swiss electricity mix (see below) and, ideally,
daily variations in usage and electricity carbon intensity.
Generic manufacturer estimates of lifetime usage-related emissions are therefore unsuitable.
There are surprisingly few sources of data on device power consumption (as opposed to CO$_2$ emissions).
Based on a smattering of data sheets, primarily from Apple, Cisco, and Dell, we model
power draw of different devices while under load using the following initial parameters:

\begin{center}
\begin{tabular}{|l|r|r|}
\hline
\textbf{Device} & \textbf{Mean (W)} & \textbf{$\sigma$ (W)} \\ \hline
Laptop & 30 & 5 \\ \hline
Desktop & 100 & 20 \\ \hline
Server & 400 & 100 \\ \hline
Phone & 5 & 2 \\ \hline
Monitor & 50 & 10 \\ \hline
Conference display & 250 & 50 \\ \hline
Printer/copier & 1000 & 200 \\ \hline
\end{tabular}
\end{center}

We assume a distribution of duty cycles and power draw when idle for each device.
In the future, we may simulate power draw at different times of day to integrate with
hourly electricity mix data, but errors in those assumptions may swamp the relatively small
hourly variations in Swiss electricity carbon intensity.

\subsection{Cloud data centers}
\label{sec:cloud_data_centers}

The secretiveness of major cloud providers makes it difficult to estimate the embodied and operational emissions of cloud
infrastructure. The best resource we have been able to find is the Datavizta API (\cite{boavizta}).

Lacking information about UZH's workloads, we model a hypothetical UZH cloud footprint as follows:
\begin{itemize}
    \item One-for-one replacement of one campus server with one Azure D8S\_v3 instance (2x Intel
      Xeon, 8 cores, 32 GB RAM, 2TB SSD) (\cite{msftvms}).
    \item Deployment to Microsoft Azure's ``\href{https://datacenters.microsoft.com/globe/explore?info=region_switzerlandnorth}{Switzerland North}'' region .
    \item Estimation of annual emissions via Datavizta, assuming constant 50\% load: 31 kg CO$_2$e for usage,
      97 kg CO$_2$e for manufacturing.
\end{itemize}

\subsection{Large language models}

Energy consumption of LLMs is being studied extensively (\cite{budennyy2022eco2ai},
\cite{castano2023exploring},
\cite{devries2023growing},
\cite{gowda2024watt},
\cite{harding2024watts},
\cite{heguerte2023estimate},
\cite{luccioni2022estimating},
\cite{luccioni2023counting},
\cite{patterson2021carbon},
\cite{rodriguez2024evaluating},
\cite{tripp2024measuring}).

The best estimates are that use of large language model (LLM) like ChatGPT averages 3-4 Wh / request,
or approximately 10x the energy of a traditional search engine query. This number appears to be decreasing
rapidly as model hardware and software improve, even though overall energy use is increasing due to
increased utilization. Nevertheless, for the model we assume that queries consume 3 Wh each.
We assume that on average every user will issue 15 queries per day.

LLMs are distributed and the location that serves any particular request is unknown, so we use the
average European electricity mix to compute carbon intensity.

The embodied emissions and training emissions of LLMs are difficult to estimate and attribute precisely.
However, \textcite{devries2023growing} provides the following information about ChatGPT:
\begin{itemize}
    \item Operational power consumption of 500 MWh / day.
    \item Approximately 1300 MWh consumed in training.
    \item Approximately 4000 servers.
\end{itemize}

Assuming a relatively clean grid at 200 g CO$_2$e / kWh (a low estimate), and that each server has embodied emissions of 4 t CO$_2$e (a high estimate),
we obtain:
\begin{itemize}
    \item 100 t CO$_2$e / day operational emissions.
    \item 260 t CO$_2$e training emissions.
    \item 16 kt CO$_2$e total embodied emissions.
\end{itemize}

After one year of operation, operational emissions are approximately 36 kt CO$_2$e, whereas training and embodied emissions are about 16 kt CO$_2$e,
so operational emissions account for $\approx 70\%$ of the total. With less conservative assumptions (dirtier grid, lower embodied emissions),
the proportion of operational emissions increases further.

As a result, when modeling overall LLM emissions, we multiply estimated operational emissions by 1.4 to account for training and embodied emissions.
We acknowledge that this is, at best, a rough estimate based on just one LLM.

\subsection{Electricity carbon intensity}

The Swiss Federal Office for the Environment published an extensive report (\cite{krebs2018umweltbilanz})
on the environmental impact of electricity generation in Switzerland in 2018.

However, for the purposes of the model, we use data from \textcite{electricitymaps}. Specifically,
we use the carbon emissions for electricity {\em consumption} (not generation) in Switzerland for 2023
at hourly granularity, then aggregate the data into two-hour blocks, and compute the mean of each two
hour block over the course of the year. This gives us twelve data points that describe average
carbon intensity variation over the course of the day.


\subsection{Limitations and validation}

The biggest limitation is the modest availability of some ground truth data about UZH IT infrastructure. Specifically, we have only estimates of campus data center energy use, and limited information abut UZH research workloads (whether AI or not). We assume that many research AI workloads will be relatively similar in intensity to existing scientific computation workloads. We also have limited information (e.g., peak power consumption) about UZH use of external high-performance compute resources such as Alps/CSCS.

Additionally, cloud vendors and AI companies are secretive about their infrastructure. We have relied on public datasets and research literature to estimate cloud and AI emissions.

Despite these limitations, our bottom-up model aligns closely with two UZH data points. First, the model correctly estimates approximately $400$ kW of compute for central campus data centers. Second, for university-owned campus IT infrastructure, the model estimates $1.4$ kt CO$_2$e emissions per year, close to the $1.45$ kt reported in the 2024 university sustainability report \cite{uzh:sustainability:report}.

\section{Results}
\label{sec:results}

Our analysis shows that manufacturing (“embodied”) emissions dominate UZH's IT carbon footprint, while operational impacts, including AI adoption, are relatively modest.

\begin{itemize}

\item Under the assumption that the vast majority (90\%) of UZH students and staff have a laptop and smartphone, manufacturing personal devices accounts for over half of UZH's IT carbon footprint. For the estimated 32,000 laptops and smartphones, embodied emissions total $\sim 2400$ t CO$_2$e annually. Usage-related emissions amount to about a tenth of that.

\item Widespread adoption of LLMs and other frontier models should have moderate impact on total emissions. If everyone at UZH made heavy use of AI (15 queries daily), this would add approximately 220 t CO2e annually, of which 160 t directly attributable to queries. While this would almost double an individual's real-time energy use, the impact on overall emissions is relatively small.

\item Embodied emissions are an important consideration for university-owned devices as well. If we exclude personal devices and LLMs, then total IT emissions shrink to $\sim 1500$ t CO$_2$e, of which $\frac{2}{3}$ comprises the embodied emissions of classroom computers, wall displays, printers, etc.

\item Data center cooling and efficiency represents a potential savings opportunity. The largest operational emissions are from data center servers and cooling (320 t). UZH data centers currently require as much energy for cooling as for computation, compared to the $6-15\%$ cooling overhead of cloud industry leaders. Moving workloads to efficient cloud providers could reduce data center emissions from $\sim 560$ t to $\sim 135$ t CO$_2$e annually, though this estimate requires further validation.

\end{itemize}

\section{Recommendations}
\label{sec:recommendations}

Do not panic about LLMs. Their emissions comprise a small fraction of total IT emissions, they will get more efficient, and attempting to curb their use is likely to be a losing battle.
Invest in infrastructure for the long term. Run awareness campaigns and adapt IT policies so that personal devices and university IT systems are used for as long as possible. Amortize manufacturing emissions over the longest period possible.
Improve the cooling efficiency of university data centers, or engage cloud vendors and academic compute centers (such as CSCS) to validate the potential benefits of a migration to cloud.



\section{Related work}
\label{sec:related:work}



\section{Conclusion}

\printbibliography

\end{document}
