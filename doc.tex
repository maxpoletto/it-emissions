\documentclass[11pt]{article}
\usepackage[margin=0.7in]{geometry}
\usepackage[T1]{fontenc}
\usepackage{libertine}
\renewcommand{\sfdefault}{fvs} % Use as sans-serif
\renewcommand{\ttdefault}{fvm} % Use as monospace
\renewcommand{\ttfamily}{\small\fontfamily{\ttdefault}\selectfont} % Slightly smaller monospace font
\newcommand{\assumption}[1]{{\bf #1}}
%\usepackage{helvet} % Helvetica font for sans-serif
%\usepackage{courier} % Courier font for monospace
\usepackage{hyperref}
\usepackage[backend=biber,style=authoryear,autocite=inline]{biblatex}
\usepackage{titlesec}
\usepackage{enumitem}

\titleformat{\section}{\sffamily\Large}{\thesection}{1em}{}
\titleformat{\subsection}{\sffamily\large}{\thesubsection}{1em}{}
\titleformat{\subsubsection}{\sffamily\normalsize}{\thesubsubsection}{1em}{}
\setlength{\parindent}{0pt}
\setlength{\parskip}{0.5em}

\addbibresource{ref.bib}

\begin{document}

\title{UZH IT \& AI Emissions Analysis}
\date{27.11.2024}

\section*{Introduction}

We are trying to answer two questions:
\begin{itemize}
    \item Are there emissions savings in moving UZH's IT infrastructure to the cloud?
    \item What is the impact of increased AI use (and, in particular, of LLMs) on UZH's emissions?
\end{itemize}

To this end, we need some UZH-specific information:
\begin{enumerate}
    \item The size of UZH's student and staff population, and a rough breakdown of
        what kind of work they do (to roughly estimate their use of IT and AI).
    \item Information about UZH's current IT infrastructure (number of servers,
        classroom equipment, etc.), and also, very importantly, the energy utilization of any
        campus data centers.
    \item An estimate of the cloud compute capacity required to host UZH's current IT requirements.
\end{enumerate}

In addition, we need more general information about the possible location of cloud data centers and
LLM deployments, the embodied emissions of different types of IT equipment, and the carbon intensity
of electricity in different regions.

Below we describe our assumptions and uncertainties regarding these factors, and then describe
our modeling approach.

\section*{Data sources and unknowns}

\cite{budennyy2022eco2ai}
\cite{castano2023exploring}
\cite{devries2023growing}
\cite{gowda2024watt}
\cite{heguerte2023estimate}
\cite{luccioni2023counting}
\cite{patterson2021carbon}
\cite{rodriguez2024evaluating}
\cite{tomlinson2024carbon}
\cite{tripp2024measuring}

\subsection*{UZH parameters}

We have access to a spreadsheet (\texttt{2024-11-13\_Berechnung\_THG\_IKT\_2024.xls}) with
counts and emissions factors for various types of IT equipment at UZH. It cites approximately
10,000 laptops and monitors, 1,000 servers, 2,400 mobile phones,  150 classroom PCs, and a few
hundred printers, copiers, and projectors.

By contrast, Wikipedia \parencite{wikiuzh} cites approximately 28,000 students and 10,000 staff
at UZH. Evidently the spreadsheet only lists university assets, not personal devices.

\assumption{In modeling overall university IT emissions as well as potential LLM use,
we assume that the entire university population has a laptop and a mobile phone.}

We do not yet have any information about the breakdown of different functions or parameters, so
\assumption{we assume that IT use is uniform across the university}. This assumption needs to be
revisited.

\subsection*{Embodied emissions}
The estimation of embodied emissions introduces additional assumptions and uncertainties into
the model. The secretiveness of major cloud providers makes it especially difficult
to estimate the embodied emissions of cloud infrastructure. Our initial proposal was to ignore
embodied emissions altogether, but that leaves a big gap.

Below are some devices and their estimated embodied emissions, from various sources.

\subsubsection*{Laptops}

\begin{table}[h]
\centering
\begin{tabular}{|r|l|}
\hline
Production \& Disposal & \\
(kg CO$_2$e) & Source \\ \hline
121 & Ecoinvent  \\ \hline
104 & \textcite{teehan2013} (low) \\ \hline
338 & \textcite{teehan2013} (high) \\ \hline
244 & \textcite{rarecoil} \\ \hline
202 & \textcite{unctadder2024} \\ \hline
\end{tabular}
\label{tab:embodied_emissions:laptops}
\end{table}

The Ecoinvent numbers appear to be outliers on the low end.

The most comprehensive data comes from \textcite{rarecoil}, which lists manufacturer
information for over 90 popular devices. Mean supply chain emissions are
244 kg CO$_2$e with a standard deviation of 128 kg. We model laptop emissions using a normal
distribution with those characteristics.

\subsubsection*{Desktops}

\begin{table}[h]
\centering
\begin{tabular}{|r|l|}
\hline
Production \& Disposal & \\
(kg CO$_2$e) & Source \\ \hline
238 & Ecoinvent \\ \hline
303 & \textcite{teehan2013} \\ \hline
403 & \textcite{unctadder2024} \\ \hline
289 & \textcite{dellpcf} \\ \hline
277 & \textcite{boavizta} \\ \hline
\end{tabular}
\label{tab:embodied_emissions:desktops}
\end{table}

Most estimates are in the range of 250 - 400 kg CO$_2$e. The most detailed data comes from
Dell's Product Carbon Footprints (\textcite{dellpcf}), which they publish for each of their products.
A sample of 15 desktops has mean supply chain emissions of 289 kg CO$_2$e with $\sigma = 80$ kg.
We model desktop emissions using a normal distribution with those characteristics.

\subsubsection*{Servers}

\begin{table}[h]
\centering
\begin{tabular}{|r|l|}
\hline
Production \& Disposal & \\
(kg CO$_2$e) & Source \\ \hline
383 & \textcite{teehan2013} \\ \hline
1252 & \textcite{davy2021} (Dell) \\ \hline
1912 & \textcite{davy2021} (EC2) \\ \hline
899 & \textcite{boavizta} (medium server)\\ \hline
\end{tabular}
\label{tab:embodied_emissions:servers}
\end{table}

\textcite{davy2021} reports data for many Dell servers and estimates for different types of EC2 instances.
The former have mean supply chain emissions of 1252 kg CO$_2$e with $\sigma = 330$ kg, the latter
1912 kg with $\sigma = 885$ kg. We model server emissions using the distribution corresponding to Dell
servers.



\subsection*{Cloud data centers}

\subsection*{Large language models}

\subsection*{Electricity carbon intensity}

\cite{krebs2018umweltbilanz}

\section*{Modeling approach}

\section*{Other considerations}

We typically think of AI use as creating additional energy demands and GHG emissions.
So far, this model does not include potential energy savings caused by AI, such as improvements
to building energy management, campus logistics, etc.

\textcite{tomlinson2024carbon} make the argument that tasks such as writing
and illustrating can be done with fewer GHG emissions by AI than by humans. They compare the
estimated emissions of an LLM performing the task to those of the average human, calculated
by multiplying per-capita annual emissions by the time ($\approx 1$h) required to write or draw.
Using this approach, AI is orders of magnitude cheaper than humans, whether one uses per-capita
emissions from the US (15 t CO$_2$e/y) or India (1.9 t). Of course, this methodology focuses on
a specific creative act: the LLM does not reduce the human's overall annual emissions.

\printbibliography
\end{document}
