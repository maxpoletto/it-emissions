\documentclass[11pt]{article}
\usepackage[margin=0.7in]{geometry}
\usepackage[T1]{fontenc}
\usepackage{libertine}
\renewcommand{\sfdefault}{fvs} % Use as sans-serif
\renewcommand{\ttdefault}{fvm} % Use as monospace
\renewcommand{\ttfamily}{\small\fontfamily{\ttdefault}\selectfont} % Slightly smaller monospace font
\newcommand{\assumption}[1]{{\bf #1}}
%\usepackage{helvet} % Helvetica font for sans-serif
%\usepackage{courier} % Courier font for monospace
\usepackage{hyperref}
\usepackage[backend=biber,style=authoryear,autocite=inline]{biblatex}
\usepackage{titlesec}
\usepackage{enumitem}
\usepackage{titling}

\titleformat{\section}{\sffamily\Large}{\thesection}{1em}{}
\titleformat{\subsection}{\sffamily\large}{\thesubsection}{1em}{}
\titleformat{\subsubsection}{\sffamily\normalsize}{\thesubsubsection}{1em}{}
\setlength{\parindent}{0pt}
\setlength{\parskip}{0.5em}

\addbibresource{ref.bib}
\title{UZH IT \& AI Emissions Analysis\vspace{-1em}}
\author{Massimiliano Poletto}
\date{\vspace{-1em}27.11.2024}

\begin{document}

\maketitle
\section{Introduction}

We are trying to answer two questions:
\begin{itemize}
    \item Are there emissions savings in moving UZH's IT infrastructure to the cloud?
    \item What is the impact of increased AI use (and, in particular, of LLMs) on UZH's emissions?
\end{itemize}

To this end, we need some UZH-specific information:
\begin{enumerate}
    \item The size of UZH's student and staff population, and a rough breakdown of
        what kind of work they do (to estimate their use of IT and AI).
    \item Information about UZH's current IT infrastructure (number of servers,
        classroom equipment, etc.), and also, very importantly, the energy utilization of any
        campus data centers.
    \item An estimate of the cloud compute capacity required to host UZH's current IT requirements.
\end{enumerate}

In addition, we need more general information about the possible location of cloud data centers and
LLM deployments, the embodied emissions of different types of IT equipment, and the carbon intensity
of electricity in different regions.

Below we describe our assumptions and uncertainties regarding these factors, and then describe
our modeling approach.

\section{Data sources and unknowns}

\subsection{UZH population}

We have access to a spreadsheet (\texttt{2024-11-13\_Berechnung\_THG\_IKT\_2024.xls}) with
counts and emissions factors for various types of IT equipment at UZH. It reports approximately
10,000 laptops and monitors, 1,000 servers, 2,400 mobile phones,  150 classroom PCs, and a few
hundred printers, copiers, and projectors.

By contrast, \textcite{wikiuzh} reports approximately 28,000 students and 10,000 staff
at UZH. Evidently the spreadsheet only lists university assets, not personal devices.

\assumption{In modeling overall university IT emissions as well as potential LLM use,
we assume that the entire university population has a laptop and a mobile phone.}

We do not yet have any information about the breakdown of different functions or parameters, so
\assumption{we assume that IT use is uniform across the university}. This assumption needs to be
revisited.

\subsection{Embodied emissions}
The estimation of embodied emissions introduces additional assumptions and uncertainties into
the model. Our initial project proposal ignored embodied emissions altogether, but that leaves
too big a gap.

Below are some devices and their estimated embodied emissions, from various sources.

\subsubsection{Laptops}

\begin{center}
\begin{tabular}{|r|l|}
\hline
\textbf{Production \& Disposal} & \\
\textbf{(kg CO$_2$e)} & \textbf{Source} \\ \hline
121 & \textcite{ecoinvent}  \\ \hline
104 & \textcite{teehan2013} (low) \\ \hline
338 & \textcite{teehan2013} (high) \\ \hline
244 & \textcite{rarecoil} \\ \hline
202 & \textcite{unctadder2024} \\ \hline
\end{tabular}
\label{tab:embodied_emissions:laptops}
\end{center}

The Ecoinvent numbers appear to be outliers on the low end.
The most comprehensive data comes from \textcite{rarecoil}, which lists manufacturer
information for over 90 popular devices. Mean supply chain emissions are
244 kg CO$_2$e with a standard deviation of 128 kg.
\assumption{We model laptop supply chain emissions using a truncated normal distribution with those characteristics.}

\subsubsection{Desktops}

\begin{center}
\begin{tabular}{|r|l|}
\hline
\textbf{Production \& Disposal} & \\
\textbf{(kg CO$_2$e)} & \textbf{Source} \\ \hline
238 & \textcite{ecoinvent} \\ \hline
303 & \textcite{teehan2013} \\ \hline
403 & \textcite{unctadder2024} \\ \hline
289 & \textcite{dellpcf} \\ \hline
277 & \textcite{boavizta} \\ \hline
\end{tabular}
\label{tab:embodied_emissions:desktops}
\end{center}

Most estimates are in the range of 250 - 400 kg CO$_2$e. The most detailed data comes from
Dell's Product Carbon Footprints (\textcite{dellpcf}), which they publish for each of their products.
A sample of 15 desktops has mean supply chain emissions of 289 kg CO$_2$e with $\sigma = 80$ kg.
\assumption{We model desktop supply chain emissions using a truncated normal distribution with those characteristics.}

\subsubsection{Servers}

\begin{center}
\begin{tabular}{|r|l|}
\hline
\textbf{Production \& Disposal} & \\
\textbf{(kg CO$_2$e)} & \textbf{Source} \\ \hline
383 & \textcite{teehan2013} \\ \hline
1252 & \textcite{davy2021} (Dell) \\ \hline
1912 & \textcite{davy2021} (EC2) \\ \hline
899 & \textcite{boavizta} (medium server)\\ \hline
\end{tabular}
\label{tab:embodied_emissions:servers}
\end{center}

\textcite{davy2021} reports data for many Dell servers and estimates for different types of EC2 instances.
The former have mean supply chain emissions of 1252 kg CO$_2$e with $\sigma = 330$ kg, the latter
1912 kg with $\sigma = 885$ kg.
\assumption{We model server supply chain emissions using a truncated normal distribution corresponding
to Dell servers.}

\subsubsection{Smartphones}

\begin{center}
\begin{tabular}{|r|l|}
\hline
\textbf{Production \& Disposal} & \\
\textbf{(kg CO$_2$e)} & \textbf{Source} \\ \hline
68 & \textcite{googlepixel8} (Pixel 8)\\ \hline
54 & \textcite{appleiphone13} (iPhone 13)\\ \hline
50 & \textcite{unctadder2024} \\ \hline
50 & \textcite{lovehagen2023} \\ \hline
\end{tabular}
\label{tab:embodied_emissions:phones}
\end{center}

\assumption{We model phone supply chain emissions using a truncated normal distribution with mean
50 kg CO$_2$e and $\sigma = 10$ kg.}

\subsubsection{Other devices}

We have found less data for other types of devices. For now we model them with averages based
on a handful of sources.

\begin{description}
    \item[Computer monitors] 344 kg CO$_2$e (\textcite{teehan2013}, \textcite{dellpcf})
    \item[Conference room displays] 753 kg CO$_2$e (scaling monitor by $(40''/27'')^2$)
    \item[Printer/copier stations] 1167 kg CO$_2$e (\textcite{ecoinvent})
    \item[Network equipment] We did not find reliable lifecycle assessments for routers and switches,
      and have no information about UZH network architecture. The vast majority ($80-95\%$) of
      lifecycle emissions of network gear stem from usage (\textcite{cisco2024}, \textcite{jacob2023}).
      For now, we model network equipment as a 5\% overhead on servers (1 router or switch for 20 servers).
      (A typical data center rack contains 42 1U servers and one switch or router.)
\end{description}

\subsection{Operational emissions}

For on-premise equipment, we would like to model the Swiss electricity mix (see below) and, ideally,
daily variations in usage and electricity carbon intensity.
Generic manufacturer estimates of lifetime usage-related emissions are therefore unsuitable.
Unfortunately, there are few consistent sources of data on device power consumption.
Based on a smattering of documents from Apple, Cisco, and Dell, we model
power draw of different devices while under load as truncated normal distributions with the
following initial parameters:

\begin{center}
\begin{tabular}{|l|r|r|}
\hline
\textbf{Device} & \textbf{Mean (W)} & \textbf{$\sigma$ (W)} \\ \hline
Laptop & 30 & 5 \\ \hline
Desktop & 100 & 20 \\ \hline
Server & 400 & 100 \\ \hline
Phone & 5 & 2 \\ \hline
Monitor & 50 & 10 \\ \hline
Conference display & 120 & 25 \\ \hline
Printer/copier & 1000 & 200 \\ \hline
\end{tabular}
\end{center}

Initially, we assume a distribution of duty cycles and power draw when idle for each device.
As a next step, we may simulate power draw at different times of day to integrate with
hourly electricity mix data, but errors in those assumptions may swamp the relatively small
hourly variations in Swiss electricity carbon intensity.

\subsection{Cloud data centers}

The secretiveness of major cloud providers makes it difficult to estimate the embodied and operational emissions of cloud
infrastructure. The best resource we have been able to find is the Datavizta API (\textcite{boavizta}).

Lacking information about UZH's workloads, we model a hypothetical UZH cloud footprint as follows:
\begin{itemize}
    \item One-for-one replacement of one campus server with one Azure D8S\_v3 instance (2x Intel
      Xeon, 8 cores, 32 GB RAM, 2TB SSD) (\cite{msftvms}).
    \item Deployment to Microsoft Azure's ``\href{https://datacenters.microsoft.com/globe/explore?info=region_switzerlandnorth}{Switzerland North}'' region .
    \item Estimation of annual emissions via Datavizta, assuming constant 50\% load: 31 kg CO$_2$e for usage,
      97 kg CO$_2$e for manufacturing.
\end{itemize}

\subsection{Large language models}

Energy consumption of LLMs is being studied extensively (\textcite{budennyy2022eco2ai},
\textcite{castano2023exploring},
\textcite{devries2023growing},
\textcite{gowda2024watt},
\textcite{harding2024watts}, 
\textcite{heguerte2023estimate},
\textcite{luccioni2022estimating},
\textcite{luccioni2023counting},
\textcite{patterson2021carbon}, 
\textcite{rodriguez2024evaluating},
\textcite{tripp2024measuring}). 

The best estimates are that use of large language model (LLM) like ChatGPT averages 3-4 Wh / request,
or approximately 10x the energy of a traditional search engine query. Indications are that this
number is decreasing as model hardware and software improve, even though overall energy use is increasing
due to increased utilization.

Nevertheless, for the model we assume that queries consume 3 Wh each, and we model per-capita daily
queries using a truncated normal distribution with mean 15 and $\sigma = 5$.

\subsection{Electricity carbon intensity}

The Swiss Federal Office for the Environment published an extensive report (\textcite{krebs2018umweltbilanz})
on the environmental impact of electricity generation in Switzerland in 2018.

However, for the purposes of the model, we use data from \textcite{electricitymaps}. Specifically,
we use the carbon emissions for electricity {\em consumption} in Switzerland for 2023 at hourly
granularity, then aggregate the data into two-hour blocks, and compute the mean of each two
hour block over the course of the year. This gives us twelve data points that describe average
carbon intensity variation over the course of the day.

\section{Modeling approach}

\section{Other considerations}

\subsection{Other IT components}

For now, we do not attempt to model changes in network infrastructure or utilization due to
move to the cloud or increased LLM use. See \textcite{jacob2023} for an overview of
network sustainability.

\subsection{Potential AI benefits}

We typically think of AI use as creating additional energy demands and GHG emissions.
So far, this model does not include potential energy savings caused by AI, such as improvements
to building energy management, campus logistics, etc.

\textcite{tomlinson2024carbon} make the argument that tasks such as writing
and illustrating can be done with fewer GHG emissions by AI than by humans. They compare the
estimated emissions of an LLM performing the task to those of the average human, calculated
by multiplying per-capita annual emissions by the time ($\approx 1$h) required to write or draw.
Using this approach, AI is orders of magnitude cheaper than humans, whether one uses per-capita
emissions from the US (15 t CO$_2$e/y) or India (1.9 t). Of course, this methodology focuses on
a specific creative act: the LLM does not reduce the human's overall annual emissions.

\printbibliography
\end{document}
